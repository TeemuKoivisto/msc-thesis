This background chapter presents a literature review of the different fields that intertwine with the topic of reviewing student submitted code using programmatic methods. The fields were chosen based on the research of related systems and approaches to the problem, some of them described in the Section \ref{sec:related-systems}. When taken out of the educational context, the problem could be framed as how to analyze collections of code and reduce their dimensionality to better understand and show their higher-level patterns. As such, the topic becomes quite large. Analysis of program code itself has been a research area from the early days of computing \cite{halstead-1972, mccabe-1976, ottenstein} with topics such as code quality, optimization and clone detection being quite popular \cite{chaiyong-2018}.

What makes it different for our educational setting of reviewing code, is the analysis of a large number of relatively short excerpts of code that all attempt to solve the same problem. Therefore, we can assume the dataset to be fairly homogeneous although within even a small exercise, the variation between the different approaches can become quite large \cite{luxton-sub-variation-2013}. A custom search engine that would automatically index the code in a compact, AST derived form with selectable algorithms to calculate the similarity could be a possible approach, as demonstrated by Codewebs \cite{codewebs}.

However, in order to build a similar system from scratch we have to understand the different parts that make up such system. One part, for example, would be knowing the available data structures to represent code. Knowing how to apply information retrieval to code would be useful for a good theoretical foundation. The manual reviewing process would be interesting to investigate, since it might help to understand how the process could be automated, as well as the current automatic reviewing approaches. Assuming that teachers would want to divide the submissions into groups based on their similarities, the available similarity detection and clustering methods would be important to review.

As it follows, these topics sum up to a large background research. As part of this research some of the methods are implemented in the new system, CodeClusters, which should demonstrate better their applicabilities. Being able to pragmatically test the approaches should provide more realistic depictions of the usability of the methods. However, this thesis is limited in its scope to trial all the proposed methods and therefore the approaches are picked based on the time required to implement them, their demonstrated results as well as flexibility. This thesis does not attempt to be a conclusive research into the optimal solution and system to this problem, but an analysis into the basics of building one.

In Section~\ref{sec:ir}, the ideas of the general information retrieval system, the search engine, are introduced and the most relevant concepts and methods that could be used for educational code search. In Section~\ref{sec:analyzing-code}, the manual reviewing process of the teachers is analyzed as well as the current automatic reviewing approaches. In Section~\ref{sec:representations}, the different representations of code are described. In Section~\ref{sec:metrics}, a short overview of quantification of code into numeric representations, metrics, is described. In Section~\ref{sec:sim-detection}, a few of the methods to calculate similarities between the submissions are presented. Lastly, in Section~\ref{sec:clustering}, the most common clustering algorithms are described and their applicability to our problem discussed.
