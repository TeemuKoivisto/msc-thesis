This chapter describes the system and its architecture as well as reviews some of the previously mentioned related systems. The main design goals for CodeClusters are to create a modular and flexible tool that allows the most primitive method, text-based search, as well as advanced modeling without tying the system to a specific approach. In practice this was approached by creating a simple search interface on top of which the other features, modeling and providing feedback, were built upon. The popularity and flexibility of search engines was theorized to make the system familiar to the users and also demonstrate a unique approach compared to other related systems.

As a general description, CodeClusters is a web application that includes search, modeling and providing of feedback to student submissions. It supports Java program code submissions and instead of one-size-fits-all model, it is designed to allow composition of the different parts of review process: search, modeling and feedback, to provide teachers greater flexibility in automating their review process. 

The most prominent part of CodeClusters is the search interface, which enables teachers to search submissions using the Solr search server. Teachers can either input strings with optional Lucene operators, add custom filters to filter the submissions by their indexed values or set facets, which allow the exploration of the indexed values and their filtering. With each functionality teachers can analyze the data for outliers or otherwise interesting patterns, after which they can provide feedback to the submissions or use the search results as the dataset for modeling. The search results are also paginated to prevent fetching large numbers of submissions at once that would incur performance issues.

To add feedback, CodeClusters uses \texttt{review} database schema which contains a feedback message shown to the students, a metadata field which is visible only to the teachers and tags, which is an array of strings to label the review. Teachers can add reviews to the submissions from the search results list by selecting either whole submissions or lines of code. Also, in bottom right corner there is a review menu that allows quick selection and selection of all the submissions, the submissions of the current page or opening an add review modal to submit a review.

The teachers can also use either the search results or the whole dataset in a model to cluster them automatically. At the moment, only n-grams model is implemented with AST tokens represented in vector space as TF-IDF matrices and cosine similarity. This is the model described in the IR and array subsections in the background chapter. The model clusters the submissions into structurally similar clusters, outputs a silhouette score, plots them on a 2-D scatter plot and lets teachers select individual clusters to analyze them or to provide feedback.

In addition to search and modeling, CodeClusters also has review flows which are a feature to store and execute search, modeling and review steps automatically. This, in theory, could further help to reduce the manual work of the teachers and the review flows can be shared amongst the users of CodeClusters. However, they still require additional development to allow their edition, deletion and automatically populating the add review form using the value of the flow's review step.

After submitting reviews, teachers can visualize them from \texttt{/reviews} page to view them in a submission-to-review grid. It shows the submissions alongside the given reviews to allow teachers to analyze which submissions have received which reviews and to modify the reviews. When teacher is satisfied with the reviews, they can accept them to set their status to \texttt{SENT} which makes them visible to students. Also, teachers can reindex and execute static analyzers to index the metrics fields to Solr from \texttt{/solr} page.
